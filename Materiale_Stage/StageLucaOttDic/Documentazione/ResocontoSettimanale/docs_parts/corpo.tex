\section{Introduzione}
\subsection{Scopo del documento}
Il presente documento intende descrivere le attività svolte durante l'intero periodo di stage. Questo
documento verrà aggiornato bi-settimanalmente, come richiesto, definendo tutte le attività svolte durante l'arco temporale preso in considerazione.
\subsection{Glossario}
In Appendice sarà allegato un glossario per disambiguare il significato dei termini tecnici.
\newpage

\section{Periodo dal 17/10 al 28/10}
In questo periodo, tempo effettivo 80 ore, sono state svolte tre macro-attività.
\begin{description}
\item[Studio libreria Raphael e tecnologia JavaScript]: come prima attività si è studiata la tecnologia JavaScript, non per le conoscenze di base, già in possesso, ma per l'applicazione di \textit{design pattern}\ atti a riprodurre i concetti di interfaccia e \textit{information hiding}, non presenti nel linguaggio di script. Tuttavia questa attività non ha prodotto i risultati progettati, si è quindi deciso di concordare con l'azienda una semplice convenzione sull'uso di questi costrutti, ovvero l'uso del carattere underscore ('\_') per segnalare i membri privati. Per l'uso delle interfacce si è stabilito di utilizzarle come un controllo sul tipo di una variabile per assicurare che rispetti determinate direttive. In seguito si è studiata la libreria Raphael permettendo di assorbire le nozioni necessarie allo svolgimento dello stage e la creazione dell'interfaccia.
\item[Analisi output del progetto di stage precedente]: per questa fase erano previste 40 ore, ma lo stagista che aveva realizzato il software era ancora presente in azienda, per cui si è potuto ridurre le tempistiche in maniera notevole, riducendo la durata a 16 ore, potendo discutere, appunto, con il creatore del software si è potuto risparmiare molto tempo.
\item[Stesura analisi dei requisiti]: Considerando le due attività precedenti si è potuto stilare il documento "Analisi dei Requisiti" che è stato approvato dal tutor aziendale.
\end{description}
Oltre a queste attività, visto il tempo risparmiato si è potuta cominciare l'attività di progettazione. Inoltre in accordo con il tutor interno, visto questo risparmio di tempo si è deciso di incrementare le funzionalità dell'interfaccia stessa, fornendo le funzionalità di zoom, una piccola tabella con le statistiche sul numero di nodi (operazioni) per giudizio e la possibilità di disegnare in due modi diversi l'albero. In futuro è possibile che vengano aggiunte altre richieste per sfruttare al massimo il tempo disponibile.

\newpage
\huge{Appendice Glossario}\\
\newline
\LARGE{\textbf{D}} \\
\line(1,0){400}\\
\newline
\textbf{Design Pattern}: soluzione generale ad un problema ricorrente, ovvero un concetto teorico da applicare a problemi che si possono presentare in fase di progettazione software.\\
\newline
\LARGE{\textbf{I}} \\
\line(1,0){400}\\
\newline
\textbf{Information Hiding}: principio informatico che prevede che il funzionamento interno di una parte di un programma o di una singola classe sia nascosta rispetto a componenti esterne.\\
\newline
\LARGE{\textbf{J}} \\
\line(1,0){400}\\
\newline
\textbf{JavaScript}: linguaggio di script orientato agli oggetti, scarsamente tipizzato. A differenza di HTML e CSS è standard ECMA.
