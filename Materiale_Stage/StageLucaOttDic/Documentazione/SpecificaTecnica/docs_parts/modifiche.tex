%%Questo file serve per scrivere lo storico delle revisioni, qui commentato trovate un esempio di come strutturare ogni singola voce revisione.

\begin{footnotesize}
\begin{longtable}{|c|c|l|p{7.5cm}|}
\hline
\textbf{DATA} & \textbf{VERSIONE} & \textbf{REDATTORI} &\textbf{MOTIVO DELLA MODIFICA}\\
\hline \hline
\endhead %Viene replicato ad ogni pagina, in cima alla tabella
\multicolumn{4}{c}{\textsc{\textbf{\large Continua sulla pagina successiva}}} %viene aggiunto alla fine della pagina di ogni tabella, esclusa l'ultima
\endfoot 
\multicolumn{4}{c}{} %vuoto altrimenti aggiunge endfoot
\endlastfoot
%<-----------------------------------------------------------------------------------------------
% DATA
%\multirow{2}*{
%	AAAA/MM/GG
%} 
%& %VERSIONE
%\multirow{2}*{
%	X.Y.Z
%}
%& % REDATTORI 
%\multirow{2}[0]{3cm}{
%        Nome Cognome 
%}
%& % MOTIVO REVISIONE
%	titolo revisione con comandi predefiniti	\newline
%	Motivo revisione.\\ \hline
%<-----------------------------------------------------------------------------------------------

% DATA
\multirow{2}*{
2011/10/20
} 
&% VERSIONE
\multirow{2}*{
	1.0.0
}
&% REDATTORI 
\multirow{2}[0]{3cm}{
Luca Fongaro
}
&% MOTIVO REVISIONE
RILASCIO VERSIONE 	\newline
Rilascio della versione 1.0\\
\hline

% DATA
\multirow{2}*{
2011/10/19
} 
&% VERSIONE
\multirow{2}*{
	0.1.0
}
&% REDATTORI 
\multirow{2}[0]{3cm}{
Luca Fongaro
}
&% MOTIVO REVISIONE
\PS	\newline
Prima stesura del documento\\
\hline
\end{longtable}


\end{footnotesize}
\newpage




